%
% Documentation for the Doubly Link List API.
%
% $Author$
% $Date$
% $Revision$
%
% Compile directive:
%  latex Linklist.tex
%  dvips -t letter Linklist.dvi -o Linklist.ps
%
\documentclass[10pt,letterpaper,titlepage]{article}

\usepackage{alltt, graphicx}

\newenvironment{lquote}{\begin{list}{}{}\item[]}{\end{list}}

\pagestyle{plain}

\begin{document}

\title{A Users Guide for the\\
       Doubly Linked List API\\
       Version 1.1.0}
\author{Carl J. Nobile\\
	cnobile@AppliedTheory.com}
\date{Created: March 28, 1999\\
	Updated: \today}
\maketitle

\pagenumbering{roman}
\section{Preface}
Writing an API for a link list came about after many years of struggling with data storage problems.  I would often write link\-list code embedded in my application, exposing all of its innards to the application.  This was a nightmare to weed through as the application grew in functionality and complexity.  Often much of the functionality that I would have liked in my application would be too difficult to implement or would be kludged in.  If more than one link list was needed my beard would thin.
\vspace{8pt}

\noindent
This manual documents the implementation and use of the Doubly Linked List API.  A brief overview of the design philosophy and how the data is abstracted will be discussed followed by a thorough explanation of the calling and return mechanism of each function.
\vspace{8pt}

\noindent
I hope it is as useful for you as it has been for me.
\vspace{8pt}
\begin{flushright}
Carl J. Nobile\\
April 1999
\end{flushright}
\pagebreak

\tableofcontents
\pagebreak

\pagenumbering{arabic}
\section{Distribution}
This Doubly Linked List can be downloaded from the following sites.  The first site below has a web page dedicated to the API.  All current releases will become available here first.
\vspace{8pt}

\noindent
http://eggwarmer.AppliedTheory.com/~cnobile
\vspace{8pt}

\noindent
You will also find the API at the following site and its mirrors.
\vspace{8pt}

\noindent
ftp://sunsite.unc.edu/pub/Linux/lib
\vspace{8pt}

\noindent
Bug reports should go to me at cnobile@AppliedTheory.com.
\pagebreak

\section{License}
\begin{center}
\Large			 The "Artistic License"\\
\vspace{8pt}
\Large				Preamble\\
\end{center}
The intent of this document is to state the conditions under which a Package may be copied, such that the Copyright Holder maintains some semblance of artistic control over the development of the package, while giving the users of the package the right to use and distribute the Package in a more-or-less customary fashion, plus the right to make reasonable modifications.
\vspace{8pt}

\noindent
Definitions:
\begin{lquote}
"Package" refers to the collection of files distributed by the\linebreak Copyright Holder, and derivatives of that collection of files created through textual modification.

"Standard Version" refers to such a Package if it has not been modified, or has been modified in accordance with the wishes of the Copyright Holder as specified below.

"Copyright Holder" is whoever is named in the copyright or copyrights for the package.

"You" is you, if you're thinking about copying or distributing this Package.

"Reasonable copying fee" is whatever you can justify on the basis of media cost, duplication charges, time of people involved, and so on.  (You will not be required to justify it to the Copyright Holder, but only to the computing community at large as a market that must bear the fee.)

"Freely Available" means that no fee is charged for the item itself, though there may be fees involved in handling the item.  It also means that recipients of the item may redistribute it under the same conditions they received it.
\end{lquote}

\noindent
1. You may make and give away verbatim copies of the source form of the Standard Version of this Package without restriction, provided that you duplicate all of the original copyright notices and associated disclaimers.
\vspace{8pt}

\noindent
2. You may apply bug fixes, portability fixes and other modifications derived from the Public Domain or from the Copyright Holder.  A Package modified in such a way shall still be considered the Standard Version.
\vspace{8pt}

\noindent
3. You may otherwise modify your copy of this Package in any way, provided that you insert a prominent notice in each changed file stating how and when you changed that file, and provided that you do at least ONE of the following:

\begin{lquote}
(a) place your modifications in the Public Domain or otherwise make them Freely Available, such as by posting said modifications to Usenet or an equivalent medium, or placing the modifications on a major archive site such as uunet.uu.net, or by allowing the Copyright Holder to include your modifications in the Standard Version of the Package.

(b) use the modified Package only within your corporation or organization.

(c) rename any non-standard executables so the names do not conflict with standard executables, which must also be provided, and provide a separate manual page for each non-standard executable that clearly documents how it differs from the Standard Version.

(d) make other distribution arrangements with the Copyright Holder.
\end{lquote}

\noindent
4. You may distribute the programs of this Package in object code or executable form, provided that you do at least ONE of the following:

\begin{lquote}
(a) distribute a Standard Version of the executables and library files, together with instructions (in the manual page or equivalent) on where to get the Standard Version.

(b) accompany the distribution with the machine-readable source of the Package with your modifications.

(c) give non-standard executables non-standard names, and clearly document the differences in manual pages (or equivalent), together with instructions on where to get the Standard Version.

(d) make other distribution arrangements with the Copyright Holder.
\end{lquote}

\noindent
5. You may charge a reasonable copying fee for any distribution of this Package.  You may charge any fee you choose for support of this Package.  You may not charge a fee for this Package itself.  However, you may distribute this Package in aggregate with other (possibly commercial) programs as part of a larger (possibly commercial) software distribution provided that you do not advertise this Package as a product of your own.  You may embed this Package's interpreter within an executable of yours (by linking); this shall be construed as a mere form of aggregation, provided that the complete Standard Version of the interpreter is so embedded.
\vspace{8pt}

\noindent
6. The scripts and library files supplied as input to or produced as output from the programs of this Package do not automatically fall under the copyright of this Package, but belong to whomever generated them, and may be sold commercially, and may be aggregated with this Package.  If such scripts or library files are aggregated with this Package via the so-called "undump" or "unexec" methods of producing a binary executable image, then distribution of such an image shall neither be construed as a distribution of this Package nor shall it fall under the restrictions of Paragraphs 3 and 4, provided that you do not represent such an executable image as a Standard Version of this Package.
\vspace{8pt}

\noindent
7. C subroutines (or comparably compiled subroutines in other languages) supplied by you and linked into this Package in order to emulate subroutines and variables of the language defined by this Package shall not be considered part of this Package, but are the equivalent of input as in Paragraph 6, provided these subroutines do not change the language in any way that would cause it to fail the regression tests for the language.
\vspace{8pt}

\noindent
8. Aggregation of this Package with a commercial distribution is always permitted provided that the use of this Package is embedded; that is, when no overt attempt is made to make this Package's interfaces visible to the end user of the commercial distribution.  Such use shall not be construed as a distribution of this Package.
\vspace{8pt}

\noindent
9. The name of the Copyright Holder may not be used to endorse or promote products derived from this software without specific prior written permission.
\vspace{2pt}

\noindent
10. THIS PACKAGE IS PROVIDED "AS IS" AND WITHOUT ANY EXPRESS OR IMPLIED WARRANTIES, INCLUDING, WITHOUT LIMITATION, THE IMPLIED WARRANTIES OF MERCHANTABILITY AND FITNESS FOR A PARTICULAR PURPOSE.

\begin{center}
The End
\end{center}
\pagebreak

\section{Introduction}
There are many goals to achieve when deciding to write an API.  The functions in the library should be reenterable, easy to include in an application, platform independent, and reasonably flexible with enough functionality to be usable.  These goals can often be contradictory; however, they are achievable with enough forethought and planning.
\vspace{8pt}

\noindent
This package is sufficiently abstracted so that the programmer will neither need to know or care how it is implemented.  At least that is the goal I have striven to achieve while writing it.
\vspace{8pt}

\noindent
Within this package is found the: source files written in C; make files for various platforms and compilers; text script which sets the environment correctly when it runs the demo program created by the make utility; README, INSTALL, and HISTORY text files; Artistic license; and the documentation in \LaTeXe form.
\vspace{8pt}

\noindent
A short overview, will follow, discussing the philosophy of how the package works including a rational of the structure and type definition use in the package.
\vspace{8pt}

\noindent
Then the library itself is broken into six groups: \textbf{initialization}, \textbf{status and state}, \textbf{pointer manipulation}, \textbf{list update}, \textbf{search}, and \textbf{input/output}.

\begin{lquote}
(a) The initialization group handles the creation, initializing, and destruction of the link list.

(b) The status and state group returns various kinds of information about the status of the link list during its operation.

(c) The pointer manipulation group allows the positioning of the current pointer to the head, tail or to an arbitrary node within the list.

(d) The list update group adds and deletes nodes.

(e) The search group returns the record information based on key data or on the absolute record position.

(f) The input/output group saves or retrieves record data to or from a disk file.
\end{lquote}

\noindent
At this writing there are 29 functions in the library, each one of which is thoroughly explained and examples given when needed.
\pagebreak

\section{Overview}
When writing tools such as this, one needs to be concerned with how it affects the entire programming environment.  One of the most important aspects of this environment is the problem concerning \emph{namespace} pollution.  To minimize this problem I have used DLL\_ as a prefix to all function names and enumerated \emph{typedef}s.
\vspace{8pt}

\noindent
It is often the case that search criteria will remain the same between queries.  As such a state table is implemented that passes the current state to the search functions.  There are two functions one to set and the other to read the state table.





\pagebreak

\section{Structures}
Most implementations of link lists allocate a single node per record and these nodes are what are linked to each other.  This type of algorithm works well when the link list is embedded in the application code, but not when implementing a link list within an API, because it cannot be made reenterant.
\vspace{8pt}

\noindent
A well written Application Programming Interface (API) requires that the functions contained within it be reenterant and also creates an environment where the code can be abstracted.  In order to take advantage of these two ideas the Doubly Linked List (hereafter referred to as the DLL) has a three level hierarchy.  The first level we will refer to as the 'Top Level Struct'.  All the global data is held by one of these structures and it is allocated once for each incident of the link list.  

%\includegraphics[width=5in]{linklistDiagram.gif}

\small
\begin{alltt}
typedef struct list
   \{
   Node           *head;         /* pointer to head record */
   Node           *tail;         /* pointer to tail record */
   Node           *current;      /* pointer to current record */
   Node           *saved;        /* pointer to stored record */
   size_t         infosize;      /* size of record incident */
   unsigned long  listsize;      /* number of records in list */
   unsigned long  current_index; /* index value of current record */
   unsigned long  save_index;    /* index value of stored record */
   DLL_Boolean    modified;      /* modified flag (TRUE or FALSE) */
   DLL_SrchOrigin search_origin; /* location a search originates from */
   DLL_SrchDir    search_dir;    /* direction the search proceeds from */
   \} List;
\end{alltt}
\normalsize
\vspace{8pt}

\noindent
At the next level is the 'Node Struct'.  This structure holds the pointer to the actual record data plus the pointers to the next and prior nodes.  It is allocated once for each record structure.

\small
\begin{alltt}
typedef struct node
   \{
   Info        *info;     /* pointer to record data */
   struct node *next;     /* pointer to next node */
   struct node *prior;    /* pointer to prior node */
   \} Node;
\end{alltt}
\normalsize
\vspace{8pt}

\noindent
The third and final level is the 'Info Struct', which holds the actual data inserted by the application.  The Info Struct is defined by the developer and is only restricted by the environment in which the application runs or is compiled in.

\small
\begin{alltt}
typedef struct your_info
   \{
   type your_data;        /* Your data goes here */
   \} YourInfo;
\end{alltt}
\normalsize
\vspace{8pt}

\noindent
There is one more structure which is not part of this hierarchy.  It is only used to return the current state of the search criteria.

\small
\begin{alltt}
typedef struct search_modes
   \{
   DLL_SrchOrigin search_origin; /* Search from head, tail, or current */
   DLL_SrchDir    search_dir;    /* Search up or down */
   \} DLL_SearchModes;
\end{alltt}
\normalsize
\pagebreak

\section{Enumerations}
I'm a firm believer that the return values of functions should be predefined \emph{typedef} enumerations.  There are two reasons for this.  The first is that many compilers will complain when a switch statement is used to test the return values of functions with one or more of the enumerated values missing thus alerting the developer to using the \emph{default} statement.   The second reason is that the \emph{typedef} name can be used as the return type of the function disallowing anything other than the enumerated values to be returned.  These are good things and should be taken advantage of.
\vspace{8pt}

\noindent
Since, that at the time of this writing Booleans are not part of the C specifications I've created my own.

\small
\begin{alltt}
typedef enum
   \{
   DLL_FALSE,
   DLL_TRUE
   \} DLL_Boolean;
\end{alltt}
\normalsize
\vspace{8pt}

\noindent
Many functions return the \emph{typedef} enumerated type DLL\_Return as shown below.

\small
\begin{alltt}
typedef enum
   \{
   DLL_NORMAL,            /* normal operation */
   DLL_MEM_ERROR,         /* malloc error */
   DLL_ZERO_INFO,         /* sizeof(Info) is zero */
   DLL_NULL_LIST,         /* List is NULL */
   DLL_NOT_FOUND,         /* Record not found */
   DLL_OPEN_ERROR,        /* Cannot open file */
   DLL_WRITE_ERROR,       /* File write error */
   DLL_READ_ERROR,        /* File read error */
   DLL_NOT_MODIFIED,      /* Unmodified list */
   DLL_NULL_FUNCTION      /* NULL function pointer */
   \} DLL_Return;
\end{alltt}
\normalsize
\vspace{8pt}

\noindent
The next two enumerations are used to determine the state of search inquiries, one is used to determine the origin and the other for the direction.  These values are passed as arguments to the \emph{DLL\_SetSearchModes()} function.

\small
\begin{alltt}
typedef enum
   \{
   DLL_ORIGIN_DEFAULT,    /* Use current origin setting */
   DLL_HEAD,              /* Set origin to head pointer */
   DLL_CURRENT,           /* Set origin to current pointer */
   DLL_TAIL               /* Set origin to tail pointer */
   \} DLL_SrchOrigin;

typedef enum
   \{
   DLL_DIRECTION_DEFAULT, /* Use current direction setting */
   DLL_DOWN,              /* Set direction to down */
   DLL_UP                 /* Set direction to up */
   \} DLL_SrchDir;
\end{alltt}
\normalsize
\vspace{8pt}

\noindent
The last enumerated type is used to determine the direction of insertion or the swapping of a record.  This structure is passed as an argument to two functions \emph{DLL\_InsertRecord()} and \emph{DLL\_SwapRecord()}.

\small
\begin{alltt}
typedef enum
   \{
   DLL_INSERT_DEFAULT,    /* Use current insert setting */
   DLL_ABOVE,             /* Insert new record ABOVE current record */
   DLL_BELOW              /* Insert new record BELOW current record */
   \} DLL_InsertDir;
\end{alltt}
\normalsize
\pagebreak

\section{Functions}
The following function calls are grouped by their general functionality, as described above.  They are written in manpage style so that I only have to document the API once.

\subsection{Initialization}
\begin{description}
\item[NAME]\quad\\
DLL\_CreateList, DLL\_InitializeList, DLL\_DestroyList, --- Initialization Functions.

\item[SYNOPSIS]
\begin{alltt}

#include <linklist.h>

List *DLL_CreateList(List **list);
DLL_Return DLL_InitializeList(List *list, size_t infosize);
void DLL_DestroyList(List **list);
\end{alltt}

\item[DESCRIPTION]\quad\\
The initialization group of functions must be used in the allocation and freeing of memory used by the link list.

 \begin{description}
 \item[DLL\_CreateList]\quad\\
 This function is called first to create the environment of the link list package.  It is passed \textbf{list} a pointer to a pointer of the \emph{Top Level Struct} type \emph{List}.  This pointer is returned both as the return value of the function and in the argument \textbf{list}.

 \item[DLL\_InitializeList]\quad\\
 After the \emph{Info Struct} has been defined this function is called to initialize the environment.  It is passed \textbf{list} the value returned from the previous function call and \textbf{infosize} the size in bytes of the \emph{Info Struct}.  It returns \textbf{DLL\_ZERO\_INFO} if \textbf{infosize} is zero; \textbf{DLL\_NULL\_LIST} if the pointer \textbf{list} is NULL; and \textbf{DLL\_NORMAL} if the initialization was successful.

 \item[DLL\_DestroyList]\quad\\
 Upon exiting the application this function when called will free all memory allocated during this instance of the list.  It is passed \textbf{list} the value returned from \emph{DLL\_CreateList} and has no return value of its own, however, the argument \textbf{list} is set to NULL.
 \end{description}

\item[EXAMPLE]
\small
\begin{verbatim}

#include <stdio.h>
#include <stdlib.h>
#include <linklist.h>

typedef struct name_addr    /* Sample data structure */
   {
   char name[30];
   char street[40];
   char city[22];
   char state[3];
   char zip[11];
   } NameAddr;

void main(void)
   {
   List *NAList = NULL;
   DLL_Return DLL_Exit;

   if(DLL_CreateList(&NAList) == NULL)
      {
      fputs("Fatal Memory error", stderr);
      exit(EXIT_FAILURE);
      }

   if((DLL_Exit = DLL_InitializeList(NAList, sizeof(NameAddr)))
    != DLL_NORMAL)
      {
      (void)(DLL_Exit == DLL_ZERO_INFO
       && fputs("Size of address record is zero.\n\n", stderr));
      (void)(DLL_Exit == DLL_NULL_LIST
       && fputs("NAList points to a NULL address.\n\n", stderr));
      exit(EXIT_FAILURE);
      }

   DoYourThingHere(NAList);

   DLL_DestroyList(&NAList);
   exit(EXIT_SUCCESS);
   }
\end{verbatim}
\normalsize

\end{description}
\pagebreak

\subsection{Status and State}
\begin{description}
\item[NAME]\quad\\
DLL\_Version(), DLL\_IsListEmpty(), DLL\_IsListFull(),\\
DLL\_GetNumberOfRecords(), DLL\_SetSearchModes(),\\
DLL\_GetSearchModes(), DLL\_GetCurrentIndex()\\
 --- Status and State Functions.

\item[SYNOPSIS]
\begin{alltt}

#include <linklist.h>

char *DLL_Version(void);
DLL_Boolean DLL_IsListEmpty(List *list);
DLL_Boolean DLL_IsListFull(List *list);
unsigned long DLL_GetNumberOfRecords(List *list);
DLL_Return DLL_SetSearchModes(List *list,
   DLL_SrchOrigin origin, DLL_SrchDir dir);
DLL_SearchModes *DLL_GetSearchModes(List *list,
   DLL_SearchModes *ssp);
unsigned long DLL_GetCurrentIndex(List *list);
\end{alltt}

\item[DESCRIPTION]\quad\\
All the functions below except \emph{DLL\_Version} take as their first argument \textbf{list} the pointer returned by \emph{DLL\_CreateList}.  These functions either return or set the status or state of some aspect of the link list.

 \begin{description}
 \item[DLL\_Version]\quad\\
 This function has no arguments and returns a string in the following format:
 \begin{alltt}
Ver: 1.1.0  May 17 1999
-------------------------------
 Developed by: Carl J. Nobile
Contributions: Charlie Buckheit
               Graham Inchley
 \end{alltt}
 \vspace{-16pt}
 \item[DLL\_IsListEmpty]\quad\\
 This function determines if the link list has any nodes defined by testing if the head and tail pointers are NULL.  It returns \textbf{DLL\_TRUE} if the list is empty and \textbf{DLL\_FALSE} if the list has valid nodes.

 \item[DLL\_IsListFull]\quad\\
 This function determines if there is enough memory to create a new \emph{Node Struct} and \emph{Info Struct} by creating and then deleting them.  It returns \textbf{DLL\_TRUE} if either of the two structures could not be allocated and \textbf{DLL\_FALSE} if the memory allocations were successful.

 \item[DLL\_GetNumberOfRecords]\quad\\
 This function returns the number of records currently in the link list by retrieving a counter value.  It returns the number of nodes allocated where a return value of zero is an empty list.

 \item[DLL\_SetSearchModes]\quad\\
 This function sets the search mode state table which is used by various function in the API.  It's second and third arguments are \textbf{origin} and \textbf{dir}.  The \textbf{origin} argument can take one of four values:

  \begin{description}
  \item[DLL\_HEAD] The origin of the search starts from the node which is at the head of the list.  This is the default value if none have been set beforehand.
  \item[DLL\_CURRENT] The origin of the search starts from the currently selected node.
  \item[DLL\_TAIL] The origin of the search starts from the node which is at the tail of the list.
  \item[DLL\_ORIGIN\_DEFAULT] The origin of the search defaults to the last set value.
  \end{description}

 The \textbf{dir} argument can take one of three values:

  \begin{description}
  \item[DLL\_DOWN] The direction of the search is from the head to the tail nodes.  This is the default value if none have been set beforehand.
  \item[DLL\_UP] The direction of the search is from the tail to the head nodes.
  \item[DLL\_DIRECTION\_DEFAULT] The direction of the search defaults to the last set value.
  \end{description}

 It returns \textbf{DLL\_NOT\_MODIFIED} if an invalid value was passed in either \textbf{origin} or \textbf{dir}.  \textbf{DLL\_NORMAL} is returned if the state table was set.

 \item[DLL\_GetSearchModes]\quad\\
 This function gets the state of the search criteria, which can either be the default values or those set by \emph{DLL\_SetSearchModes}.  It's second argument is \textbf{ssp} a pointer to the structure below.  It returns a pointer to this same instance of the structure.

 \begin{alltt}
typedef struct search_modes
   \{
   DLL_SrchOrigin search_origin;
   DLL_SrchDir    search_dir;
   \} DLL_SearchModes;
 \end{alltt}

\emph{NOTE: This function has a different argument list starting with release linlkist-1.1.0.  The original function allocated the structure internally to the function which was not thread safe.}

 \item[DLL\_GetCurrentIndex]\quad\\
 This function returns the index of the current record by retrieving a counter value.  A return value of zero is an empty list.
 \end{description}

\item[EXAMPLE]\quad\\
Examples of most of these functions can be seen in the source file \emph{dll\_test.c} used in the testing of the link list API.

\end{description}
\pagebreak

\subsection{Pointer Manipulation}
\begin{description}
\item[NAME]\quad\\
DLL\_CurrentPointerToHead(), DLL\_CurrentPointerToTail(),\\
DLL\_IncrementCurrentPointer(), DLL\_DecrementCurrentPointer(),\\
DLL\_StoreCurrentPointer(), DLL\_RestoreCurrentPointer()\\
  --- Pointer Manipulation Functions.

\item[SYNOPSIS]
\begin{alltt}

#include <linklist.h>

DLL_Return DLL_CurrentPointerToHead(List *list);
DLL_Return DLL_CurrentPointerToTail(List *list);
DLL_Return DLL_IncrementCurrentPointer(List *list);
DLL_Return DLL_DecrementCurrentPointer(List *list);
DLL_Return DLL_StoreCurrentPointer(List *list);
DLL_Return DLL_RestoreCurrentPointer(List *list);
\end{alltt}

\item[DESCRIPTION]\quad\\
The \emph{current} pointer in the link list keeps track of the last used node.  In order for this to be of benefit there needs to be a way of controlling where this pointer is located within the list.  These functions allow the repositioning and storing of this pointer during program execution.
\vspace{8pt}

\noindent
All of these functions return the enumerated type \emph{DLL\_Return} and take only one argument \textbf{list} the pointer returned by \emph{DLL\_CreateList}.

 \begin{description}
 \item[DLL\_CurrentPointerToHead]\quad\\
 This function sets the \emph{current} pointer to the head of the list and sets the \emph{index} counter to one.  A return value of \textbf{DLL\_NULL\_LIST} indicates that the list has no nodes allocated and \textbf{DLL\_NORMAL} indicates that the function succeeded in its task.

 \item[DLL\_CurrentPointerToTail]\quad\\
 This function sets the \emph{current} pointer to the tail of the list and sets the index counter to the \textbf{listsize} counter.  A return value of \textbf{DLL\_NULL\_LIST} indicates that the list has no allocated nodes and \textbf{DLL\_NORMAL} indicates that the function succeeded in its task.

 \item[DLL\_IncrementCurrentPointer]\quad\\
 This function increments the \emph{current} pointer and the \emph{index} counter each by one.  A return value of \textbf{DLL\_NULL\_LIST} indicates that the list has no allocated nodes, \textbf{DLL\_NOT\_FOUND} indicates that the end of the list has been reached, and \textbf{DLL\_NORMAL} indicates that the function succeeded in its task.

 \item[DLL\_DecrementCurrentPointer]\quad\\
 This function decrements the \emph{current} pointer and the \emph{index} counter each by one.  A return value of \textbf{DLL\_NULL\_LIST} indicates that the list has no allocated nodes, \textbf{DLL\_NOT\_FOUND} indicates that the beginning of the list has been reached, and \textbf{DLL\_NORMAL} indicates that the function succeeded in its task.

 \item[DLL\_StoreCurrentPointer]\quad\\
 This function stores the \emph{current} pointer and the \emph{index} counter in the \emph{Top Level Struct} for later retrieval.  Only one value can be stored at a time so calling this function again will destroy the first stored pointer and index values.  A return value of \textbf{DLL\_NOT\_FOUND} indicates that the list is empty and \textbf{DLL\_NORMAL} indicates that the function succeeded in its task.

 \item[DLL\_RestoreCurrentPointer]\quad\\
 This function restores the \emph{current} pointer and the \emph{index} counter from the \emph{Top Level Struct}.  Since only one value can be stored at a time calling this function again will return the last pointer and index values.  A return value of \textbf{DLL\_NOT\_FOUND} indicates that the list is empty and \textbf{DLL\_NORMAL} indicates that the function succeeded in its task.
 \end{description}

\item[EXAMPLE]\quad\\
Examples of most of these functions can be seen in the source file \emph{dll\_test.c} used in the testing of the link list API.

\end{description}
\pagebreak

\subsection{List Update}
\begin{description}
\item[NAME]\quad\\
DLL\_AddRecord(), DLL\_InsertRecord(), DLL\_SwapRecord(),\\
DLL\_UpdateCurrentRecord(), DLL\_DeleteCurrentRecord(),\\
DLL\_DeleteEntireList() --- List Update Functions.

\item[SYNOPSIS]
\begin{alltt}

#include <linklist.h>

DLL_Return DLL_AddRecord(List *list, Info *info,
   int (*pFun)(Info *, Info *));
DLL_Return DLL_InsertRecord(List *list, Info *info,
   DLL_InsertDir dir);
DLL_Return DLL_SwapRecord(List *list, DLL_InsertDir dir);
DLL_Return DLL_UpdateCurrentRecord(List *list,
   Info *record);
DLL_Return DLL_DeleteCurrentRecord(List *list);
DLL_Return DLL_DeleteEntireList(List *list);
\end{alltt}

\item[DESCRIPTION]\quad\\




\end{description}
\pagebreak


\end{document}
